

\documentclass[17pt]{extarticle} % set to 12 when done

\usepackage{outlines}
\usepackage{graphicx} % Add pictures to your document
\usepackage{listings} % Source code formatting and highlighting
\usepackage{amsmath}
\usepackage{amsthm}
\usepackage{extsizes}
\usepackage{amsfonts} 

\usepackage[margin=0.5in]{geometry}




\begin{document}
\newcommand\abs[1]{\left|#1\right|} % absolute
\newcommand{\seq}[2][0]{\left\{ #2 \right\}_{n=#1}}
% ^ newcommand, use like \seq[startingg_n]{sequence}
\newtheorem{theorem}{Theorem}
\newtheorem*{theorem*}{Theorem}
\newtheorem*{definition}{Definition}

\section{Chapter 11, Sequences}




Sequences are a kind of summation, written in the following form:

$$
    \seq[1]{n \over n+1}^\infty
$$


They can be alternating, meaning that $x_n$ is on the opposite side of $x_{n+1}$ for any n. Sequences are basically an ordered list of computed values, kinda like a function, but discrete instead of possibly continuous. Sequences can also depend on the previous term, which would make them recursive. A popular sequence like this is the fibionacci sequence:

$$
    \begin{aligned}
        x_1 & = 0                          \\
        x_2 & = 1                          \\
        x_n & = \seq[1]{x_{n-1} + x_{n+2}} \\
        \text{which gives}                 \\
        \{0, 1, 1, 2, 3, 5, 8, \ldots\}
    \end{aligned}
$$


A sequence $\seq{a_n}$ has a limit L and we write
$$
    \begin{aligned}
        \lim_{n \to \infty} \seq{a_n} = L  \\
        or                                 \\
        a_n \to L \text{ as } n \to \infty \\
    \end{aligned}
$$
If a sequence has a limit L, it can be said it converges to L, else it diverges.
If $\seq{a_n}$ converges to L, then each $a_{n+1}$ is closer to L than $a_{n}$


\begin{theorem*}[Th 5]
    $$
        \lim_{n\to \infty} = \infty
    $$
    For every positive number $M$, there is an integer N such that if $n >  N$, then $a_n > M$
\end{theorem*}

Example 1:
$\seq{2n}, M=1000, N = 500$
Dan is er voor elke $M > 1000, N \ge 500$



\begin{theorem*}[Th 6]
    If $\seq{a_n}, \seq{b_n}$  are convergent sequences and c is a constant, then the following limit laws apply
\end{theorem*}

$$
    \begin{aligned}
        \lim_{n \to \infty} a_n \pm b_m       & =  \lim_{n \to \infty} a_n \pm \lim_{n \to \infty} b_n                          \\
        \lim_{n \to \infty} c a_n             & = c \lim_{n \to \infty} a_n                                                     \\
        \lim_{n \to \infty}  ( a_n \cdot b_n) & = \lim_{n \to \infty}  a_n \cdot \lim_{n \to \infty}  b_n                       \\
        \lim_{n \to \infty}  \frac{a_n}{b_n}  & = {\lim_{n \to \infty}  a_n \over \lim_{n \to \infty } b_n} if\ \lim b_n \neq 0 \\
        \lim_{n \to \infty}  (a_n)^p          & = \left( \lim_{n \to \infty}  a_n \right)^p \text{if n and p } > 0
    \end{aligned}
$$

All the normal limit laws apply as well, so review ana1.

\begin{theorem*}[Squeeze th.]
    Squeeze theorem also applies to sequences. If $a_n \leq b_n \leq c_n$ and $\lim a_n = \lim c_n = L$,
    then $\lim b_n$ = L
\end{theorem*}

\begin{theorem*}[Th 6]
    If lim $\abs{a_n}$ = 0, then lim $a_n$ = 0
\end{theorem*}

See examples in notebook.

\begin{theorem*}[Th 7]
    If $\lim a_n = L$, and if $f(x)$ is continuous at $L$, then lim $f(a_n) = f(L)$

\end{theorem*}

\begin{theorem*}
    $\seq{a_n}$ is increasing if $a_n < a_{n+1}$ for all $n > 0$,\\
    is decreasing if $a_n < a_{n-1}$ for all $n > 0$,\\
    is monotonic if it's only decreasing or increasing

\end{theorem*}

\begin{definition}
    $\seq{a_n}$ is bounded ${above \over below}$ if there is a number $M$ such that $a_n$ is always ${below \over above} M$ for all n
\end{definition}

\begin{theorem*}[Th 12, Monotonic sequence theorem]
    Every bounded monotone sequence is convergent

\end{theorem*}


\newpage
\section{Chapter 11.2, Sequences}

A series is just a summation of a sequence. An infinite series is a summation of an  infinite sequence, usually noted in the following forms:
$$
\sum^\infty_{n=1}a_n\ or\ \sum a_n
$$

Furthermore, we can also have partial sums, denoted as $S_n$. This is defined as 
$$
S_n = \sum^n_{n=1} = a_1 + a_2 + \ldots + a_n
$$

If given a series $S_n$, we can compute $a_n$ by subtracting $S_n$ with $S_{n-1}$. \\
Example:

$$
\begin{aligned}
    S_n &= {n+1 \over n+10} \\
S_{n-1} &= {n \over n+9} \\
a_n &= S_n - S_{n-1}  \\
&=  {n+1 \over n+10} - {n \over n+9} \\
&= {9 \over (n +9) (n + 10)}
\end{aligned}
$$


We can also compute $a_k$ for any integer constant k, given an $S_n$ by subtracting $S_k$ by $S_{k-1}$. \\
Example: 
Calculate $a_7$ for the given $S_n$:
$$
\begin{aligned}
    S_n &= {n + 1 \over n + 10} \\
    a_7 &= S_7 - S_6 = {8\over 17} - {7 \over 16} = {9\over 272}
\end{aligned}
$$

\begin{theorem*}
    Given a series $S_n = \sum a_n$, we can compute if it diverges or converges, and where it does so 
    by computing the limit of $s_n$ as n goes to infinity. Or more math-y: 
    $$
    \lim_{n\to \infty} S_n = L\\
    $$
    if $L \neq \infty , S_n $ converges to L.

\end{theorem*}


\subsection{Geometric series}
Geometric series are a special kind of series written in the following form:
$$
\sum^\infty_{n=0} a(r)^k
$$
Example: 
$$
S_n = 8 + {8 \over 3} + {8 \over 9} + {8 \over 27} + \ldots \\
S_n = 8 ({1\over 3})^n
$$

The special property of these is that for any $r \in <-1, 1>$, 
$$
\sum^\infty a(r)^k = {a \over 1 - r}
$$
If r $\not\in <-1, 1>$, then $S_n$ diverges.

\begin{theorem*}   
    if $\sum a_n, \sum b_n$ are convergent, then so are the following:
    $$
    \begin{aligned}
        \sum c_an &= c \sum a_n \\
        \sum \left( a_n + b_n \right) &= \sum a_n + \sum b_n \\
    \end{aligned}
    $$
\end{theorem*} 

\subsection{Divergence/convergence tests}
There are several tests to see whether a series diverges or not. 
The first one is the n'th term test. This one does not guarantee that if a series passes this test, that it converges.
But it does guarantee that if it fails the test, that it diverges.

\begin{theorem*}{Divergence test} \\
    If
    $$
    \lim_{n \to \infty} a_n \neq 0, 
     \sum^\infty a_n 
    $$ 
    will diverge.Furthermore, if $\sum a_n$ converges, $lim a_n = 0$
\end{theorem*}
This makes sense, because if $a_n$ diverges to a nonzero value, as n approaches infinity, you'd be summing up that value infinity times, 
which does reach infinity. And if there is no limit, it obviously can't converge.

\begin{theorem*}{P-series test}\\
    A series in the form $\sum {1\over n^p}$ is called a p-series, and converges for all $p > 1$
\end{theorem*}

\begin{theorem*}{Integral test}\\
    If f(n) is a cont, decreasing function on [1, $\infty$], then $\sum^\infty_{n=k} f(n)$ converges 
    iff $\int_k^\infty f(x)dx$ converges(returns a real, non infinite number)\\
    Furthermore, if $R_n = S - S_n$, 
if given a continuous, decreasing function f(n) and $\sum f(n)$ converges, 
then 
$$
\int^\infty_{n+1} f(n) dn \leq R_n \leq \int^\infty_n f(n) dn
$$ 
    
\end{theorem*}

\begin{theorem*}{Comparison test} \\
Given an $\sum a_n \text{ and } \sum b_n$ where $a_n, b_n > 0$ and $a_n < b_n$, if $b_n$ converges, $a_n$ converges too. \\
Conversely, if $a_n$ diverges, so does $b_n$.\\

\end{theorem*}

\begin{theorem*}{Limit comparison test}\\
    Given an $a_n, b_n$,  and both are positive for all $n \in \mathbb{N}$\\
    if $\lim_{n \to \infty} {a_n \over b_n} = \text{L where } L \in \mathbb{R^+}$, then they both converge or diverge. 
    
\end{theorem*}  

\begin{theorem*}{Alternating series test}\\
    Given a series $\sum a_n$ where $a_n = (-1)^n \cdot b_n \text{ or } (-1)^{n +1} \cdot b_n$ \\
     where $b_n > 0$ and $\lim b_n = 0$ and $b_n$ is decreasing, $\sum a_n$ converges 
    
\end{theorem*}
\newpage
\begin{theorem*}{Ratio test}\\
    Given a series $\sum a_n$, if 
    $$
    \begin{aligned}
        \lim \abs{a_{n+1} \over a_n} = L \\
        then if \\
        L < 1 \to \text{series converges}\\
        L > 1 \to \text{series diverges}\\
        L = 1 \to \text{inconclusive}\\
    \end{aligned}
    $$
    
\end{theorem*}

\subsection*{Conditional and absolute convergence}
Given a series $\sum a_n$, \\
if it converges but $\sum \abs{a_n}$ diverges, we say that it \textbf{converges conditionally}. \\
If $\sum \abs{a_n}$ converges too, we speak of \textbf{absolute convergence}.
\end{document}









