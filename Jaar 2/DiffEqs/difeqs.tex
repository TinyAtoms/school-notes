

\documentclass[17pt]{extarticle} % set to 12 when done

\usepackage{outlines}
\usepackage{graphicx} % Add pictures to your document
\usepackage{listings} % Source code formatting and highlighting
\usepackage{amsmath}
\usepackage{amsthm}
\usepackage{extsizes}

\usepackage[margin=0.5in]{geometry}




\begin{document}
\newcommand\abs[1]{\left|#1\right|} % absolute
\newcommand{\seq}[2][0]{\left\{ #2 \right\}_{n=#1}}
% ^ newcommand, use like \seq[startingg_n]{sequence}
\newtheorem{theorem}{Theorem}
\newtheorem*{theorem*}{Theorem}
\newtheorem*{definition}{Definition}

\section{Chapter 1, differential equations}

Differential equations(DEs) are equations in the form of 
$$
y'' + 2ty' + sin(t)y = ln(t)
$$
Basically, an equation that has one or more deratives in it.

we can use the following conventions to write the derative:
$$
\begin{aligned}
    y' = y \\
y'(x) = y(x) \\
y'(t) = y(t) \\
{dy\over dt} = y(t)
\end{aligned}
$$

DEs have an order, and the order is the highest derative.
Also note that the n'th derative for n>2 is written like $y^{(n)}$
an example would be the following:
$a y^{(5)} + b y^{(3)} + y'' + 2y = y$, where the order is 5.

Differential equations can be linear or not. 
to be linear, a function has to be in the following form: 
$$
ay'' + by' = cy
$$
where a,b,c can be any sin, cos, tan, ln, etc of any variable. As long as the y part is linear, meaning no
ln(y), sin(y), $y^2$, etc

\newpage
\section{Solving DEs}
First, we have to determine the type of DE. There are lots, but here are the basic ones:

\subsection{Seperable DEs}
If we can split the DE in the following form:
$$
(ay + by ){dy \over dx} = (at + bt)
$$
Basically, if you can seperate the y's and the t's to different sides.
The strategy is then to just integrate both sides to  y or t (depending on the side)

\subsection{Linear DEs}
If the equation is or can be written in the standard form 
$$
{dy\over dt} + p(t) y= g(t)
$$

Then we can multiply both sides by $u(t)$, where u(t) is the following:  

$$
\begin{aligned}
u(t) &= e^{\int p(t)dt} \\
\text{Which should deliver} \\
u(t) {dy\over dt} + u(t) p(t) y = u(t) g(t)   
\end{aligned}
$$
If done correctly, then the following holds true:
$$
u(t) {dy\over dt} + u(t) p(t) y = [p(t) y]'
$$
The next step is to integrate the right side to t, and bring u(t) y back. 
like this:
$$
u(t) y = \int u(t) g(t) dt
$$
The final step is to make it an explicit function of y if asked, and to determine what C is if given a starting value.

\subsection{Homogenous DEs}
If the equation can be written in the form of 
$$
{dy \over dx} = f({y \over x})
$$

This is most often done by factoring out an $x^n$
We then substitute every \textbf{${y\over x}$} with \textbf{v}, and \textbf{y'} with \textbf{v' x + v}
We then make it into a seperable DE with the Vs on one side, and the Xs on the other, integrate both sides and then substitute the v back with ${y \over x}$
\subsection{Exact DEs}
A DE is exact if we can put it in the following form:
$$
\begin{aligned}
    M dx + N dy &= 0\\
    \text{where } \frac{\partial M}{\partial y} &= {\partial N \over \partial x}
\end{aligned}
$$

When that's done, 
*NOTE, ASK JOSH FOR CORRECT NOTES*
\end{document}









